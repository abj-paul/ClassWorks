% Created 2022-10-02 রবি 15:51
% Intended LaTeX compiler: pdflatex
\documentclass[11pt]{article}
\usepackage[utf8]{inputenc}
\usepackage[T1]{fontenc}
\usepackage{graphicx}
\usepackage{longtable}
\usepackage{wrapfig}
\usepackage{rotating}
\usepackage[normalem]{ulem}
\usepackage{amsmath}
\usepackage{amssymb}
\usepackage{capt-of}
\usepackage{hyperref}
\author{Abhijit Paul, 1201}
\date{\textit{<2022-10-02 রবি>}}
\title{Analyzing websites using Nmap Tool}
\hypersetup{
 pdfauthor={Abhijit Paul, 1201},
 pdftitle={Analyzing websites using Nmap Tool},
 pdfkeywords={},
 pdfsubject={},
 pdfcreator={Emacs 27.1 (Org mode 9.5.5)}, 
 pdflang={English}}
\begin{document}

\maketitle
\clearpage \tableofcontents \clearpage

\section{Result of the Analysis}
\label{sec:orgfe929b3}
We will note down our important findings in the beginning because results essentially evaluates an experiment or analysis task.\\
\begin{enumerate}
\item SSL Certificate of almost all gov.bd websites will expire in 40 days. As almost all of the .gob.bd subdomains shares a single SSL Certificate of bangavhaban.gov.bd. However, eprocurement have its own SSL-Cert.\\
\item dhakasouthcity.gov.bd DNS entries are VERY inconsistent. The site or ip has been perhaps moved away to another site called addwire.com but dns query or reverse dns query gives results that shows that it is a functioning subdomain of .gov.bd.\\
\item eprocure.gov.bd is the most robust and secured .gov.bd website we have found.\\
\item Most govt servers have almost similar structure of services and ports open.\\
\item Most govt servers have almost similar version of operating system.\\
\item They also fall under same IP Block. \texttt{(103.163.210.0-255)}\\
\item All websites uses BDCCL datacenter.\\
\item All govt sites have Linux 2.6 OS running but Linux 2.6 versions have reached their End-of-life long ago (by 2015) according to \href{https://en.wikipedia.org/wiki/Linux\_kernel\_version\_history}{Wikipedia} .It may be a system level vulnerability.\\
\item 30 Vulnerabilities were found in dhakasouthcity.gov.bd, 25 of which are due to old SSH version.\\
\end{enumerate}
\section{Target Specification}
\label{sec:orge0912ae}
The following sites were used to perform nmap analysis.\\
\begin{center}
\begin{tabular}{lrl}
Sitename & IP Address & About\\
\hline
eprocure.gov.bd & 103.40.82.19 & Electronic Tender\\
moi.gov.bd & 103.163.210.117 & Ministry of Information\\
dhakasouthcity.gov.bd & 170.10.162.208 & Dhaka South City Corporation\\
shed.gov.bd & 103.163.210.130 & Secondary \& Higher Secondady Education\\
bdpost.gov.bd & 103.163.210.131 & Digital Post Office\\
\end{tabular}
\end{center}
\section{Scan Techniques}
\label{sec:org23fb1eb}
\subsection{Scanning TCP Ports}
\label{sec:org74333dc}
For moi.gov.bd, the following TCP ports were found.\\
\begin{center}
\begin{tabular}{lll}
PORT & STATE & SERVICE\\
\hline
111/tcp & open & rpcbind\\
143/tcp & open & imap\\
443/tcp & open & https\\
465/tcp & open & smtps\\
587/tcp & open & submission\\
646/tcp & filtered & ldp\\
993/tcp & open & imaps\\
995/tcp & open & pop3s\\
2222/tcp & open & EtherNetIP-1\\
3306/tcp & open & mysql\\
\end{tabular}
\end{center}

For eprocure.gov.bd, the following TCP ports were found.\\
\begin{center}
\begin{tabular}{lll}
PORT & STATE & SERVICE\\
\hline
80/tcp & open & http\\
443/tcp & open & https\\
\end{tabular}
\end{center}

For shed.gov.bd, the following TCP ports were found.\\
\begin{center}
\begin{tabular}{lll}
PORT & STATE & SERVICE\\
\hline
80/tcp & open & http\\
389/tcp & closed & ldap\\
443/tcp & open & https\\
1503/tcp & closed & imtc-mcs\\
1719/tcp & closed & h323gatestat\\
1720/tcp & closed & h323q931\\
2000/tcp & closed & cisco-sccp\\
5060/tcp & closed & sip\\
\end{tabular}
\end{center}

For dhakasouthcity.gov.bd, the following TCP ports were found.\\
\begin{center}
\begin{tabular}{lll}
PORT & STATE & SERVICE\\
\hline
21/tcp & open & ftp\\
22/tcp & filtered & ssh\\
25/tcp & open & smtp\\
26/tcp & open & rsftp\\
53/tcp & open & domain\\
80/tcp & open & http\\
110/tcp & open & pop3\\
111/tcp & open & rpcbind\\
143/tcp & open & imap\\
443/tcp & open & https\\
465/tcp & open & smtps\\
587/tcp & open & submission\\
646/tcp & filtered & ldp\\
993/tcp & open & imaps\\
995/tcp & open & pop3s\\
2222/tcp & open & EtherNetIP-1\\
3306/tcp & open & mysql\\
\end{tabular}
\end{center}

For bdpost.gov.bd, the following TCP ports were found.\\
\begin{center}
\begin{tabular}{lll}
PORT & STATE & SERVICE\\
\hline
80/tcp & open & http\\
389/tcp & closed & ldap\\
443/tcp & open & https\\
1503/tcp & closed & imtc-mcs\\
1719/tcp & closed & h323gatestat\\
1720/tcp & closed & h323q931\\
2000/tcp & closed & cisco-sccp\\
5060/tcp & closed & sip\\
\end{tabular}
\end{center}

\subsection{Scanning Ports using ACK packet}
\label{sec:orgcaa018b}
It is a different port scanning as it does not find out whether a port is open or closed. Rather it finds out if the firewall filteres the packets for that port.\\
It sends a empty ACK packet. If the packet goes through firewall and reaches the TCP port, they will return RST packet. If the firewall drops the packet, then no response will arrive. Thus, we can easily identify whether a port is filtered or unfiltered (firewall enabled or not).\\

We found the following results from TCP ACK scan.\\
\begin{enumerate}
\item moi,bdpost,shed has all ports unfiltered except for http and https.\\
\item eprocure has all ports filtered.\\
\item dhakasouthcity wensite is a special case.2 of its ports(ssh,ldp) are filtered. All other ports are unflitered.\\
\end{enumerate}

\subsubsection{Analysis of Output}
\label{sec:org28d4705}
We did notice one inconsistency here. That is, bdpost.gov.bd ACK scan contains \texttt{sip-service} unfiltered while SYN scan does not contain this port. We then looked into why this happened.\\

Is it due to random chances of unpredictable network condition? So we ran the experiment for the site again.\\
\begin{center}
\begin{tabular}{lll}
PORT & STATE & SERVICE\\
\hline
80/tcp & open & http\\
389/tcp & closed & ldap\\
443/tcp & open & https\\
1503/tcp & closed & imtc-mcs\\
1719/tcp & closed & h323gatestat\\
1720/tcp & closed & h323q931\\
2000/tcp & closed & cisco-sccp\\
5060/tcp & closed & sip\\
\end{tabular}
\end{center}
We can see that it was indeed due to random chances. The new TCP SYN scan output contains sip-service.\\
\subsection{Scanning Ports using SYN packet/Stealth Scan}
\label{sec:org05aea21}
We get the \textbf{\textbf{\textbf{same output}}} as TCP Port scan, naturally.\\
However, we do need to understand that the difference between TCP port scan and SYN scan is that, TCP port scan establishes a connection with the port while TCP Syn only sends SYN packet and receives the SYN-ACK to confirm the TCP port is open. So it is possible to scan thousands of ports per second using this method.\\
It is called stealth scan because it never completes a full TCP connection. Because of the stealth and fastness of this method, it is the most popular TCP port scanning method.\\
\subsection{Scanning UDP Ports}
\label{sec:orgbcc5f2b}
For eprocure.gov.bd, the following UDP ports were found.\\
\begin{verbatim}
All 1000 scanned ports on eprocure.gov.bd (103.40.82.19) are open|filtered
\end{verbatim}

For moi.gov.bd, the following UDP ports were found.\\
\begin{center}
\begin{tabular}{lll}
PORT & STATE & SERVICE\\
\hline
389/udp & filtered & ldap\\
1701/udp & filtered & L2TP\\
1719/udp & filtered & h323gatestat\\
2000/udp & filtered & cisco-sccp\\
5060/udp & filtered & sip\\
\end{tabular}
\end{center}

For dhakasouthcity.gov.bd, the following UDP ports were found.\\
\begin{center}
\begin{tabular}{lll}
PORT & STATE & SERVICE\\
\hline
111/udp & open & rpcbind\\
137/udp & closed & netbios-ns\\
161/udp & open & snmp\\
177/udp & closed & xdmcp\\
427/udp & closed & svrloc\\
500/udp & closed & isakmp\\
520/udp & closed & route\\
623/udp & closed & asf-rmcp\\
626/udp & closed & serialnumberd\\
1645/udp & closed & radius\\
1812/udp & closed & radius\\
2049/udp & closed & nfs\\
5353/udp & closed & zeroconf\\
10080/udp & closed & amanda\\
17185/udp & closed & wdbrpc\\
\end{tabular}
\end{center}

For shed.gov.bd, the following UDP ports were found.\\
\begin{center}
\begin{tabular}{lll}
PORT & STATE & SERVICE\\
\hline
389/udp & filtered & ldap\\
1701/udp & filtered & L2TP\\
1719/udp & filtered & h323gatestat\\
2000/udp & filtered & cisco-sccp\\
5060/udp & filtered & sip\\
\end{tabular}
\end{center}

For bdpost.gov.bd, the following UDP ports were found.\\
\begin{center}
\begin{tabular}{lll}
PORT & STATE & SERVICE\\
\hline
389/udp & filtered & ldap\\
1701/udp & filtered & L2TP\\
1719/udp & filtered & h323gatestat\\
2000/udp & filtered & cisco-sccp\\
5060/udp & filtered & sip\\
\end{tabular}
\end{center}
\subsubsection{Further Analysis}
\label{sec:orgfc7ffbc}
We found \texttt{OPEN|FILTERED} response after scanning UDP ports in eprocure.gov.bd This is a very interesting output. Each UDP protocol has different packet format and nmap sends empty packets for most services. As a result, the UDP ports will drop the packet. It means, the port is open as no SMTP error message were returned. Or at least it was true in early internet. Nowadays, firewall are used in every server and the firewall can also drop packets. So there is no way to verify whether the packet drop was due to firewall or UDP service. So in 2004, a new version of nmap came out and it defined this new output: OPEN or FILTERED.\\

This issue can be managed if we send service-specific package instead of empty packets. The UDP port will reply to the packet in this case and thus, we can identify open ports from filtered ports. The service scanning command of nmap (\texttt{nmap -sUV -F felix.nmap.org}) sends service-specific packets by default. So we will use that to check if eprocure.gov.bd has any open UDP port.\\
\section{Domain Analysis}
\label{sec:orge1c656d}
\begin{center}
\begin{tabular}{lrll}
Sitename & Domain IP & Extra IP & WhoIS\\
\hline
moi.gov.bd & 103.163.210.121 & 103.163.210.117 & BDCCL\\
bdpost.gov.bd & 103.163.210.131 & None & N/A\\
dhakasouthcity.gov.bd & 170.10.162.208 & None & N/A\\
eprocure.gov.bd & 103.40.82.19 & None & N/A\\
shed.gov.bd & 103.163.210.130 & None & N/A\\
\end{tabular}
\end{center}

For dhakasouthcity, \textbf{\textbf{\textbf{reverse DNS lookup}}} gave the following output.\\
\begin{verbatim}
170.10.162.208: addwire.com
\end{verbatim}
\section{Port Specification}
\label{sec:org5c1a68b}
We have already found open ports list from Scan Techniques Section. Now we will elaborate on what each of those ports do.\\
\begin{itemize}
\item ftp: File Transfer Protocol Service\\
\item ssh: To remotely access server.\\
\item smtp: Mail Transfer Protocol\\
\item rsftp: Tools for communicating using SSH File Transfer Protocol(SFTP)\\
\item domain: Authentication \& Authorization in local network.\\
\item http: Presenting webresource to interent.\\
\item pop3: Old Mail Transfer Protocol\\
\item rpcbind: Necessary for windows process communication between devices.\\
\item imap: Combined with SMTP or pop3, it allows you to read emails.\\
\item https: Secured HTTP\\
\item smtps: SMTP with SSL or TLS cryptographic protocol.\\
\item submission: User AGent for email\\
\item ldp: Protocol to switch between protocols that router uses.\\
\item imaps: IMAP but secured with TLS-SSL layers\\
\item pop3s: POP3 but secured with TLS-SSL layers\\
\item mysql: Database Management Service\\
\end{itemize}

\section{Service \& Version Detection}
\label{sec:orgf64a93e}
For bdpost.gov.bd, moi.gov.bd and shed.gov.bd, we were able to find the exact service software of only two service. We were not able to approximate their versions.\\

\begin{center}
\begin{tabular}{llll}
PORT & STATE & SERVICE & VERSION\\
\hline
80/tcp & open & http & nginx\\
443/tcp & open & ssl/http & nginx\\
\end{tabular}
\end{center}

For eprocure.gov.bd, we were able to find the exact service-version software of two services.\\
\begin{center}
\begin{tabular}{llll}
PORT & STATE & SERVICE & VERSION\\
\hline
80/tcp & open & http-proxy & F5 BIG-IP load balancer http proxy\\
443/tcp & open & ssl/http & Apache httpd (JSP/2.3)\\
\end{tabular}
\end{center}

For dhakasouthcity.gov.bd, we were able to find exact service-version of Many services.\\
\begin{center}
\begin{tabular}{llll}
PORT & STATE & SERVICE & VERSION\\
\hline
21/tcp & open & ftp & Pure-FTPd\\
22/tcp & filtered & ssh & \\
26/tcp & open & smtp & Exim smtpd 4.95\\
53/tcp & open & domain & PowerDNS Authoritative Server 4.4.1\\
80/tcp & open & http & LiteSpeed\\
110/tcp & open & pop3 & Dovecot pop3d\\
111/tcp & open & rpcbind & 2-4 (RPC \#100000)\\
143/tcp & open & imap & Dovecot imapd\\
443/tcp & open & ssl & /https LiteSpeed\\
465/tcp & open & ssl & /smtp  Exim smtpd 4.95\\
587/tcp & open & smtp & Exim smtpd 4.95\\
646/tcp & filtered & ldp & \\
995/tcp & open & pop3s & ?\\
2222/tcp & open & ssh & OpenSSH 7.4 (protocol 2.0)\\
3306/tcp & open & mysql & MySQL 5.7.39-cll-lve\\
\end{tabular}
\end{center}

\section{OS Detection}
\label{sec:org264f5eb}
\begin{center}
\begin{tabular}{ll}
Website & OS Guess\\
\hline
shed.gov.bd & Linux 2.6.18 - 2.6.22 (89\%)\\
moi.gov.bd & Linux 2.6.18 - 2.6.22 (89\%)\\
eprocure.gov.bd & Linux 2.6.18 - 2.6.22 (97\%)\\
dhakasouthcity.gov.bd & Linux 3.10 - 4.11 (95\%)\\
bdpost.gov.bd & Linux 2.6.18 - 2.6.22 (89\%)\\
\end{tabular}
\end{center}
\subsection{Analysis}
\label{sec:orgf1a648b}
Linux 2.6 versions have reached their End-of-life long ago (by 2015) according to \href{https://en.wikipedia.org/wiki/Linux\_kernel\_version\_history}{Wikipedia} .It may be a system level vulnerability.\\
\section{NSE Scripts}
\label{sec:orgcb0a600}
\subsection{Default Script}
\label{sec:orgaf1477d}
The default script returned the OS version, service list, SSL-Certificate and version that we have already seen. So the outputs won't be recorded in the document but they will be in the experimentaiton folder for the respective sites.\\
\subsection{Sitemap Generator Script}
\label{sec:org56503cc}
For bdpost.gov.bd,dhakasouthcity.gov.bd,eprocure.gov.bd,moi.gov.bd and shed.gov.bd,  no sitemap was found. The system administrators have hidden the structure in their apache configuration file.\\
\begin{verbatim}
PORT     STATE  SERVICE
80/tcp   open   http
| http-sitemap-generator: 
|   Directory structure:
|   Longest directory structure:
|     Depth: 0
|     Dir: /
|   Total files found (by extension):
|_    
443/tcp  open   https
| http-sitemap-generator: 
|   Directory structure:
|     /
|       Other: 1
|   Longest directory structure:
|     Depth: 0
|     Dir: /
|   Total files found (by extension):
|_    Other: 1
\end{verbatim}
\subsection{DNS-Brute script}
\label{sec:orgadd3a9d}
DNS records hold a surprising amount of host information. By brute forcing them we can reveal additional targets.\\

For shed.gov.bd,bdpost.gov.bd,dhakasouthcity.gov.bd,eprocure.gov.bd,domain.gov.bd, we got the following outputs.\\
\begin{verbatim}
Host script results:
| dns-brute: 
|   DNS Brute-force hostnames: 
|     testing.gov.bd - 123.49.12.132
|     ntp.gov.bd - 103.163.246.78
|_    lab.gov.bd - 103.163.210.131
\end{verbatim}
\subsubsection{What are these Testing, Ntp, Lab servers?}
\label{sec:org751afad}
Using \texttt{whois} command, we find that-\\
\begin{enumerate}
\item lab.gov.bd belongs to BDCCL. Perhaps it is their lab for govt website as BDCCL seems to be hosting almost all govt websites.\\
\item ntp.gov.bd belongs to Optimus Technology who has a Tier 3 datacenter in bangladesh.\\
\item testing.gov.bd brings an interesting result. Perhaps, as APINCC gives IP to all of asia pacific, this IP block happened to land to a chinese company but we are not too sure about this.\\
\end{enumerate}
\begin{verbatim}
China Telecom
descr: No.31,jingrong street
\end{verbatim}
\section{IDS Evasion}
\label{sec:org4c7c019}
Surprisingly, all IDS script using \texttt{-D RND} came out empty. Is it due to random chances? So we ran the command again and indeed, it was due to random chances.\\
\begin{verbatim}
abhijit@abhijit-H81M-S2PV:~$ sudo nmap -D RND -sA shed.gov.bd
Starting Nmap 7.80 ( https://nmap.org ) at 2022-10-02 14:25 +06
Nmap scan report for shed.gov.bd (103.163.210.130)
Host is up (0.10s latency).
Not shown: 994 filtered ports
PORT     STATE      SERVICE
389/tcp  unfiltered ldap
1503/tcp unfiltered imtc-mcs
1719/tcp unfiltered h323gatestat
1720/tcp unfiltered h323q931
2000/tcp unfiltered cisco-sccp
5060/tcp unfiltered sip

Nmap done: 1 IP address (1 host up) scanned in 9.36 seconds#+end_src
\end{verbatim}
Perhaps as the commands were run at night, it was hard to find correct Random ip address to spoof for nmap.\\
\end{document}